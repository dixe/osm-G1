    \documentclass[11pt]{article}
\usepackage[utf8]{inputenc}
\usepackage[danish]{babel}
\usepackage{fancyhdr}
\usepackage{icomma}
\usepackage[final]{pdfpages}
\setcounter{secnumdepth}{-1}
\usepackage{amsfonts}
\usepackage{float}
\pagestyle{fancy}
\usepackage{url}
\usepackage{graphicx}
\usepackage{fancyvrb}
\usepackage{alltt}
\usepackage{listings}

\lstset{
    breaklines     = true,
    numbers        = left,
    stepnumber     = 1,
    numberstyle=\color{black},
    showstringspaces=false,
    language = C,
    frame=single,
    basicstyle=\ttfamily\scriptsize\color{red!40!black},
    keywordstyle=\bfseries\color{blue},
    commentstyle=\color{green!40!black},
    identifierstyle=\ttfamily\color{black},
    stringstyle=\color{yellow!65!black},
}


\renewcommand{\headrulewidth}{0.0pt} %sætter så der ikke er en linje%
\renewcommand{\footrulewidth}{0.0pt} %sætter så der ikke er en linje%
\lhead{}
\chead{}
\rhead{}
\begin{document}

\title{G1 rapport}
\author{Sebastian Ostenfeldt Jensen,\\ Thomas Wolff Rosenqvist and\\ Nikolaj Østergaard}
\date{17. February 2014}
\maketitle
\thispagestyle{empty}
\pagebreak
\pagenumbering{arabic}
\setcounter{page}{1}

\section{Task 1}
Task 1 contains three files \verb|double_linked_list.h|, \verb|double_linked_list.c| and \verb|main.c|. The two first files are the implementation of the double linked list using xor, \verb|main.c| contains the testing of the task. To insert in O(1) time at either head or tail, we first check whether current head are NULL, if it is, we just insert the new node as head and tail, and set the pointers to NULL. We choose not to let the list be cyclic. Head always points to the next node, and tails points the previous node. If we insert at head and head != NULL, we first update current head's pointer, by xoring the current pointer, which points to the next node, with the new node we are about to insert. After updating, we set the next node's pointer to the current head, and then we set the list's head to the new node. Inserting at the tail is almost the same.
\newline
When extraction we get a new head or tail, and that is dependent on the node we extracted, so when extraction, we need to reset that pointer so it does not depend on the node we just removed. We do so by xor'ing the new head/tail's pointer with the node we just extracted, and set that to new new head/tail pointer.
\newline
To reverse the list, we just switch around head and tail, since head points to the next node, and tail points to previous, when switch around it is the same.
\newline
Searching in is just testing if the head is the item we are looking for, if it is we return the item it points to, if not we xor it with the previous node, to get the next node, and test that. We do this until we are at the tail, and if we have not found what we where lokking for, we return NULL.
\newline
We have added two functions \verb|init_dlist| which initialize a new empty list, and \verb|free_dlist| which extract each node, and free its memory.
\section{Task 2}

\subsection{syscall write}
The way we implemented the \verb|syscall_write|, was by adding a function with the name \verb|sycall_write| inside \verb|proc/syscall.c|. And in the same file we added a case more in the function \verb|syscall_handle|, that called our \verb|syscall_write| with register A1, A2 and A3, and saved the result in V0. Below the code for \verb|syscal_write| is shown in chunks
\newline
\begin{lstlisting}
device_t *dev;
gcd_t *gcd;
int len = 0;
/*
* we always write to terminal, and we don't need fhandle
* assign it to itself to avoid compiler warnings ie errors
*/
fhandle = fhandle;
\end{lstlisting}
First of we \verb|device_t dev| to hold the console, the we create a \verb|gcd_t gcd| to do the writing at last we set fhandle to it self, because we don't need it, and if we didn't we would get unused variable error.
\newline
\begin{lstlisting}
/*Find the system console (first tty)) */
/* Should be FILEHANDLE_STDOUT, which is 1, but when
* using 1 and not 0 i get kernel assert failed on dev
*/
dev = device_get(YAMS_TYPECODE_TTY,0);
if(dev == NULL){
  return -1;
}
/* Set generic char device*/
gcd = ( gcd_t *) dev->generic_device;
if(gcd == NULL){
  return -1;
}
\end{lstlisting}
In the code above on line 5 we set the device to stdin, as stated in the code comment, when setting it to stdout i.e. 1, the code would return -1, when setting to 0, it would write to console.
\newline
\begin{lstlisting}
len = gcd->write(gcd, buffer, length);

return len;
\end{lstlisting}
When the gcd has been set, we use it to write to the screen

\subsection{syscall read}
The read syscall is implemented in the same way as write, the only difference is that instead of calling \verb|syscall_write| we call \verb|syscall_read|, the arguments are the same and we also save the return value in register V0. The code is shown below.
The first 20 lines are the same as write, where we initialize gcd and makes sure it is not NULL, after that we reach the code below. 
\begin{lstlisting}
while(len <= length && !( *(char*) (buffer+len-1)== 13)){
    /* we read one byte at a type, so we increment len, and
     * store the next byte on the offset of len
     */
    len += gcd->read(gcd, buffer + len, length);
    /* If we don't hit enter, write to
     * the screen what the user typed in
     */
    if(!( *(char*) (buffer+len-1)== 13)){
      // should only write one char
      syscall_write(0,buffer+len-1,1);
    }
  }
\end{lstlisting}
We decided to implement read, so that it could read more the one byte per syscall, the while loop runs until we reach the maximum length we are allowed to read, or when the user presses enter. In ascii 13 is carriage return, which is enter. This implementation can easily be change, so that it also stops on EOF, newline or whatever is preferred. The reason we choose enter for now, is because we only read from console, where it is normal to press enter when done typing. One line 11 we echo the byte the user just entered. We only want to echo if the user didn't press enter. After exiting the loop we just return len, code not shown.

\subsection{readwrite test}
Testing of read and write is just two buffers, one that starts out by writing to the screen and asking for input. Now it is a test of read, it will read until we have read 31 bytes, or the user presses enter. We then echo what the user wrote and ask to write again. The reason to write again, is that read echo every char that the  types, using write syscall, with at length argument of 1. By reading twice, we are sure that the second time, the buffer we read into are not full on '\\0', and in that way we test that write works correctly. If it does, it just echo the char the user types, if not, it writes the  rest of the buffer. Our testing of it found that both read and write works as intended.
\end{document}


